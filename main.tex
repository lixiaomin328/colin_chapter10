\documentclass{article}
\usepackage[utf8]{inputenc}

\title{chapter10}
\author{lixiaominxiaomin }
\date{March 2021}

\begin{document}

\maketitle
I:  Theoretical neuroeconomics 
a.	Self-control
a.	Fudenberg, benhabib, brocas
b.	Memory and “time travel”
a.	brocas
c.	States and cues
a.	Laibson bernheim rangel 
d.	Individual differences: ASD and ADHD
a.	Landry  mention Haunshofer
II. Low-hanging fruit 
a.	Advertising etc. What part of neuroscience is necessary,, and why? 
b.	Addiction What part of neuroscience is necessary,, and why? 
c.	Emotion and psych game theory . What part of neuroscience is necessary,, and why? 
d.	Emotion and skills in labor markets. What part of neuroscience is necessary,, and why? 
e.	Incommensurability, repugnance etc 
f.	Mens rea (Vilares et al PNAS, Jones et al Mens Rea paper)
g.	Changing behavior through changing brain activity  write a  section for Chapter 10 about causality. Ballesta OFC results.  Lusha Zhu sleep results.  Ming Hsu, Benedetto lesion results.  Figner? Tms and impatience results.  Simpler results from instruction (Nikki Sullivan?) 
i.	
III. What’s next? 


labor market autism  attention\_advantage\_ofautism\_Neuropsych\_Lavie.pdf   (better performance on inattential blindness and no degradation from cognitive load) – lady cop in The Bridge. Tv show sweden
	This concluding chapter has two purposes. The first purpose to describe several analyses which illustrate theoretical neuroeconomics. These technical descriptions first appear here, rather than in previous chapters, because of my belief that is better to hear all about the brain first then use that knowledge to both judge and appreciate the modelling, rather than the other way around.  

	The second purpose of this concluding chapter is to sketch some important topics in social science which are most likely to benefit from neuroscientific constructs, facts, and methods. For some of these a natural next step is creating neuroeconomic theory, using the models in the first part of the chapter as exemplars. This section is like  a grant proposal.  

I:  Theoretical neureconomics
Theoretical neureconomics is the creation of mathematical theories of choice which are inspired by neuroscientific facts. This is an unusual enterprise. In most areas of economic theory, the foundations of a new theory are inherited from previous theories and use the same modelling conventions and style. Decision theories often take the form of systems of axioms which are shown to uniquely imply a particular behavior or functional form describing behavior. The functional form describes behavior if and only if the behavior satisfies those axioms. Such axioms are not justified as biological principles, with some rare exceptions. \footnote{The axiomatic treatment of reward prediction error in Rutledge et al CHECK, and evidence in the form of axiom testing, is one such rare exception. We discussed that in Chapter 2.}
Theoretical neuroeconomics should be judged by different standards, in my opinion. Intuitive plausibility of axioms is not enough. Instead, good neuroeconomic theory should have two “must have” features:
1.	A good theory should make falsifiable predictions that can be tested by biological data and are particularly interesting as claims about neural processes. 
2.	A good theory should be fruitful and improvable, in sense that other scientists can read and understand it in order to create generalizations or add different features with plausible added predictability.  
Besides those features, there are some “would be nice to have” features of theory too. For example, a theory that makes predictions across species, or across the human life cycle of ontogeny, is especially useful. Such theories expand the set of available and collectible data to test theory.
In addition, many theories have the desirable property that conventional rational choice is nested inside the theory as a special case, when one or more parameters take on specific values. An example is the neural autopilot theory of habit from chapter 2: When the reward reliability threshold $\sigma$ (below which habit-mode is engaged) is zero, the model collapses to conventional rational choice. As has been so successful in behavioral economics (Rabin AER P\&P), this type of parametric nestedness means that every parameter measurement is a test of the relative strength of conventional and neuroeconomic explanations. 
It’s even better if the theory can be constrained so it is easily comparable to alternative behavioral theories (as in the Benhabib and Bisin (2005) endogeneous-control model of consumption and savings). It’s also useful if the theory has a natural causal interpretation, since there are many methods for direct causal influence on brain regions, especially for non-human animals. 
And finally, it is not necessary, but is potentially useful, if a neuroeconomic theory can say something about human welfare. In plain language, consumer welfare what is best for consumers as they themselves would judge it. A welfare analysis allows us to analyze whether proposed policies are good or bad, for welfare thus defined. This is a messy topic, even for behavioral economics, and is no less messy for neuroeconomics. The scientists who have thought hardest about the welfare consequences of different behavioral theories do not always agree. Welfare judgments based on neuroeconomics will be harder still. 
We touched on the topic of welfare in the chapter 2 discussion of “wanting” and “liking”. Liking should be the welfare criterion. Decoupling of wanting and liking therefore implies that people are not always wanting-- and choosing-- what is best for them. That implies there is room for improvement in their own choices (as judged by their own liking, not by a bureaucrat, religious figure, parent, psychiatrist, advice columnist, or Twitter user). Whether a neuroeconomic theory can generate interesting welfare analyses and defensible policies remains to be seen. It is not a major goal of our enterprise. 
Cowen paper ??
I.	Neuroeconomic theory 
a.	






a.	Self-control
Fudenberg and Levine (AER 2006) 
Fudenberg and Levine (2006) present a dual-self model of impulse control. A long-run self (LR) makes decisions which can constrain the series of  decisions of myopic  short-run selves (S\_t). The constraints create a self-control cost. The long-run self’s utility is the discounted sum of the short-run utilities minus self-control costs. 
It is a simple and appealing approach. I think of it as metaphorically as describing a parent and the behavior of a child who is myopic. 


Here are the moving parts, for the special case of consumption-savings decisions: $y$ is a state variable (such as income); $a$ is the savings action by a self $S_t$; ($1-a$ is consumption); and $r$ is a self-control action chosen by the LR-self. In the savings example, income in period t+1 is $y_{t+1}=Ra_ty_t$ where R is the rate of return on savings and $a_t$ is the percentage of money saved. (Consumption everything and saving nothing is $a_t=0.$) 
The S sel(ves) in general have myopic utility $u(y,r,a)$ The “base” or unconstrained preference of the S-self is given by 
 
This utility is decreasing in $a$; left to his own devices, the myopic self would spend everything by choosing $a=0$. This is not good from the point of view of the LR-self, because it reduces future income because $y_{t+1}=Ra_ty_t$.
Some precursor assumptions imply a cost of self-control which is a linear multiple of the difference between the S-self utility from consuming nothing and the base preference, as follows:
 
For a special case of constant relative risk-aversion  (CRRA) utility, there is a unique solution to the LR-self’s optimization problem, which is a constant savings rate $a$. This reduces the LR-self’s reduced form utility (implicitly controlling for choices of self-control $r$) of  

$U_{RF}$
 
Recall that the LR-self can choose a constraining value of $a^*$ which the $S_t$ selves will be constrained to obey. Differentiating the $U_{RF}$ with respect to $a$ gives the first-order condition 
 
The constrained $a^*$ that maximizes LR-self utility is decreasing in $\lambda$, showing that more costly self-control leads to less constraint on S-self saving. Less obviously, it is also increasing in the discount factor $\delta$, so that a more patient LR-self allows more S-self consumption. 

\section{Introduction}

\end{document}
